\dev{Greg Depoire}

\section{Intro informelle}

\subsection{Contexte}

\subsubsection{Intervenants}

\begin{itemize}
	\item Laurent Feuilloley (LIRIS)
	\item Théo Pierron (LIRIS)
	\item Stéphan Tomassé (LIP) qui remplace Nicolas Bousquet
\end{itemize}

Poser plutôt les questions à Laurent Feuilloley et Théo Pierron.

\subsubsection{MCC}

\begin{itemize}
	\item Examen sur table
	\item \emph{Scribing notes} : $\text{un cours} \to \text{un scribe} + \text{un comité éditorial}$
\end{itemize}

\subsubsection{Le modèle étudié}

\begin{itemize}
	\item Le graphe en entrée est le réseau. Les nœuds sont les unités de calcul et les arêtes sont les liens de communication.
	\item Chaque nœud commence avec la seule connaissance de lui-même et de ses liens. Il doit terminer avec une partie de la solution.
	
	\begin{exemple}[Coloration]
		Chaque nœud termine avec une couleur différente de ses voisins.
	\end{exemple}
	
	\begin{exemple}[Arbre couvrant]
		Chaque nœud termine avec un parent (voisin ou lui même si racine) tel que le total forme un arbre.
	\end{exemple}
	\item \emph{Breaking symmetry} : dans certains graphes comme les cycles, les nœuds sont équivalents. On donne donc à chaque nœud $x$ un identifiant unique $i(x) \in [n^2] = \{ 1, \dots, n^2 \}$ où $n$ est le nombre de nœuds du graphe.
\end{itemize}

\subsubsection{Rounds de communication}

Initialement, chaque nœud connaît son identifiant et ses liens. Lors d'un round, les nœuds envoient et reçoivent un message de leurs voisins puis effectuent un calcul. À la fin du dernier round, chaque nœud termine avec sa solution locale.

\subsubsection{Complexité}

La complexité est le nombre de rounds effectués pour calculer une solution. Le temps de calcul est négligé devant les communications.

\subsubsection{Localité}

Après $k$ rounds, un nœud connaît son voisinnage à distance $k$. Pour que l'algorithme soit efficace, il faut donc prendre la décision en utilisant une petite partie locale de l'instance (\og fast = local \fg).

\subsection{Un mauvais cas}

Supposons que l'on veuille effectuer une tâche simple de colorer un chemin de $n$ nœuds.

\begin{center}
	\begin{tikzpicture}
		
	\end{tikzpicture}
\end{center}

Il n'existe pas d'algorithme distribué qui $2$-colore $P_{n + 3}$ en moins de $n / 2$ rounds.

\begin{preuve}
	Supposons qu'un tel algorithme existe.
	
	L'algorithme colorie $x$ et $y$ sur la connaissance de leur voisinnage à distance $< n / 2$.
	
	Ainsi, en l'appliquant sur sur le cycle de longueur $n + 3$ on obtient une $2$-coloration d'un cycle impair ce qui est impossible. L'algorithme n'existe donc pas.
\end{preuve}

\subsection{3-colorer un chemin}

On suppose maintenant que le réseau est le chemin \emph{orienté} de longueur $n$. Chaque nœud peut toujours communiquer avec ses deux voisins, mais il y a une distinction entre le voisin de gauche et le voisin de droite.

\subsubsection{Algo naïf}

\begin{complexite}
	$\Theta(n^2)$
\end{complexite}

\subsubsection{Algo bruteforce}

Il y a $n - 1$ rounds de communication. Chaque nœud se colorie avec la parité de la distance à la racine. On obtient une $2$-coloration.

\begin{complexite}
	$\Theta(n)$
\end{complexite}

\subsubsection{Réduction des couleurs}

Si un nœud $x$ a une couleur plus grande que ses voisins, on le recolorie en $1$, $2$ ou $3$ en fonction de ses voisins.

Il existe une méthode en $log^*(n)$.

\subsubsection{Cole-Vishkin}

L'algorithme de Cole-Vishkin permet de réduire une $C$-coloration en une 6-coloration en $\log^* C$ rounds.

\begin{note}
	$\log^* n = \text{nombre de } \lfloor \log \rfloor \text{ à appliquer pour atteidre } 1$
	
	C'est la fonction réciproque de $\tour(n) = \underbrace{2^{2^{.^{.^2}}}}_{n \text{ fois}}$.
\end{note}

On note $x$ le nœud qui exécute l'algorithme, $y$ son nœud voisin et $c(x)$ la couleur du nœud $x$.

On obtient une $6$-coloration.

\begin{correction}
	Premièrement, l'invariant \og $c(x) != c(y)$ \fg est vrai au début et est préservé à chaque round donc la coloration retournée est bien propre.
	
	Deuxièmement, si toutes les couleurs sont inférieures à $C$ au début d'un round, alors elles sont inférieures à $2 \lceil \log C \rceil$ à la fin du round. On peut prouver par induction qu'au bout de $\log^* n^2$ rounds, on obtient une $6$-coloration.
\end{correction}

\begin{note}
	$6$ est un point fixe de $2 \lceil \log \rceil$.
	
	\bigskip
	\begin{center}
		\begin{tabular}{ |c|c| } 
			\hline $n$ & $2 \lceil \log n \rceil$ \\ 
			\hline $9$ & $7$ \\ 
			\hline $8$ & $6$ \\
			\hline $7$ & $6$ \\
			\hline $6$ & $6$ \\
			\hline
		\end{tabular}
	\end{center}
\end{note}

\begin{complexite}
	$O(\log^* n)$
\end{complexite}

\begin{note}
	Pour les chemins non orientés, on calcule $c(x)$ à partir des couples $(p, b)$ correspondants aux deux voisins.
\end{note}

Pour $3$-colorier le chemin, on peut à présent appliquer la réduction des couleurs 3 fois.

\subsection{Nombre chromatique et line graph}

\begin{definition}
	Le nombre chromatique $\chi(G)$ d'un graphe $G$ est le nombre minimal de couleurs nécéssaire à la coloration de $G$.
\end{definition}

On peut construire des graphes de nombre chromatique arbitrairement grand.

\begin{exemple}
	La clique $K_n$ vérifie $\chi(K_n) = n$.
\end{exemple}

Et si on interdit les triangles (3 sommets reliés les uns aux autres) ?

\begin{exemple}[Propositions des élèves]
\end{exemple}

\begin{exemple}[Mycielski]
\end{exemple}

\subsubsection{Pour plus tard}

\begin{definition}
	La \emph{girth} d'un graphe est la longeur d'un plus court cycle du graphe.
\end{definition}

\begin{note}
	Si un graphe a pour girth $2k + 2$ alors pour chaque sommet $x$, le graphe des sommets à distance $k$ de $x$ est un arbre.
\end{note}

\begin{theoreme}[Erdös]
	Il existe des graphes de grand nombre chromatique et de girth arbitraire.
\end{theoreme}

\subsubsection[Autre modèle : le shift graph]{Autre modèle : le \emph{shift graph}}

\begin{definition}
	$\shift_n$ est le graphe orienté dont les nœuds sont les couples $i j$ avec $1 \le i < j \le n$ et les arêtes sont les couples $i j \to j k$.
\end{definition}

\begin{exemple}[$\shift_4$]
\end{exemple}

Il n'y a pas de triangle dans les shift graphs.

Colorer les nœuds de $\shift_n$ équivaut à colorer les arêtes de $K_n$.

Colorer les nœuds de $shift_n$ équivaut à colorer les arêtes de $K_n$.

\begin{definition}
	$\ramsey_k(3)$ est le $n$ minimum tel que toute $k$-coloration des arêtes de $K_n$ contient un triangle monochromatique.
\end{definition}

\begin{lm}{}{}
	\[ \ramsey_2(3) = 6 \]
\end{lm}

\begin{preuve}
	Il existe une $2$-coloration des arêtes $K_5$ sans triangle monochromatique.
	
	Cepandant, une coloration des arêtes de $K_6$ en 2 couleurs contient forcément un triangle monochromatique.
	
	Soit un sommet $x$. Il existe une couleur majoritaire $c$ parmi les arêtes de $x$. On considère les voisins correspondants aux arrêtes de cette couleur $c$.
	
	Si toutes les arêtes entre eux sont de l'autre couleur alors il un a un triangle monochromatique.
	
	Sinon, il existe une arête de couleur $c$ entre eux et il y a aussi un triangle monochromatique.
\end{preuve}

La notion de \emph{line graph} généralise les shift graphs.

\begin{definition}[Line graph]
	Soit $G$ un graphe orienté. Le \emph{line graph} de $G$ est le graphe orienté $L(G)$ défini par :
	
	\begin{itemize}
		\item Les sommets sont les arêtes de $G$.
		\item Les arêtes sont les couples $i j arrow.r j k$ où $i$, $j$ et $k$ sont des sommets de $G$.
	\end{itemize}
\end{definition}

\begin{exemple}
	\[ \shift_n = L(\overrightarrow{K_n}) \] où $\overrightarrow{K_n}$ est le graphe à $n$ sommets dans lequel $i \to j$ ssi. $i < j$.
\end{exemple}

On veut comparer $\chi(L(G))$ à $\chi(G)$.

\begin{lemme}
	\[ \log \chi(G) \le \chi(L(G)) \]
\end{lemme}

\begin{preuve}
	Considérons une $k$-coloration de $L(G)$. On transforme cette coloration en une coloration de $G$. Si $x$ est un sommet de $G$ alors on pose \[ c'(x) = \{ c(x y) | x \to y \} \]
	
	$c$ est propre car si $x \to y$ alors $c(x y) \in c'(x)$ mais $c(x y) \not\in c'(y)$. En effet, si il existait un $z$ tel que $c(x y) = c(y z) \in c'(y)$, alors alors la coloration de $L(G)$ ne serait pas propre car $x y \to y z$ dans $L(G)$.
	
	La coloration de $G$ obtenue est une $2^k$ coloration donc $\chi(G) \le 2^{\chi(L(G))}$.
\end{preuve}

